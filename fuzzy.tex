\documentclass[12pt]{article}
\usepackage{listings}
\usepackage{xcolor}
\usepackage{hyperref}

\lstdefinestyle{mystyle}{
    backgroundcolor=\color{gray!10},
    basicstyle=\ttfamily\small,
    breakatwhitespace=false,
    breaklines=true,
    captionpos=b,
    keepspaces=true,
    numbers=left,
    numbersep=5pt,
    numberstyle=\tiny\color{gray},
    showspaces=false,
    showstringspaces=false,
    showtabs=false,
    tabsize=2,
    frame=single
}

\lstset{style=mystyle}

\title{Soft Computing: Fuzzy and Crisp Composition, and Fuzzy Equivalence}
\author{Dhruba Saha \\
Roll no: B.Sc(Sem-VI)Comp-05 \\
B.Sc(Honors) 3rd year 6th sem \\
Department of Computer \& System Sciences \\
Siksha Bhavana \\
Visva Bharati}
\date{}

\begin{document}

\maketitle

\newpage
\section*{Dependencies}
The following dependencies are required for the implementation of the assignment:

\begin{itemize}
    \item Python 3.x
    \item NumPy library
\end{itemize}

\section*{Github Repository}
\url{https://github.com/dhrubasaha08/DSE-4-Soft-Computing-Lab}

\newpage
\section{Fuzzy composition using max-min}

\lstinputlisting[language=Python, caption=Fuzzy max-min composition]{fuzzy_max_min_composition.py}

\newpage
\begin{lstlisting}[caption=Output of Fuzzy max-min composition]
Enter dimensions of matrix A (n, m): 2 2
Enter row 1: 0.4 0.6
Enter row 2: 0.7 0.3
Enter dimensions of matrix B (m, p): 2 2
Enter row 1: 0.6 0.9
Enter row 2: 0.5 0.6
Matrix A:
 [[0.4 0.6]
 [0.7 0.3]]
Matrix B:
 [[0.6 0.9]
 [0.5 0.6]]
Fuzzy max-min composition:
 [[0.5 0.6]
 [0.6 0.7]]
\end{lstlisting}

\newpage
\section{Fuzzy composition using max-product}

\lstinputlisting[language=Python, caption=Fuzzy max-product composition]{fuzzy_max_product_composition.py}

\newpage
\begin{lstlisting}[caption=Output of Fuzzy max-product composition]
Enter dimensions of matrix A (n, m): 2 2
Enter row 1: 0.7 0.5
Enter row 2: 0.6 0.1
Enter dimensions of matrix B (m, p): 2 2
Enter row 1: 0.6 0.7
Enter row 2: 0.1 0.1
Matrix A:
 [[0.7 0.5]
 [0.6 0.1]]
Matrix B:
 [[0.6 0.7]
 [0.1 0.1]]
Fuzzy max-product composition:
 [[0.42 0.49]
 [0.36 0.42]]
\end{lstlisting}

\newpage
\section{Crisp composition using max-min}

\lstinputlisting[language=Python, caption=Crisp max-min composition]{crisp_max_min_composition.py}

\newpage
\begin{lstlisting}[caption=Output of Crisp max-min composition]
Enter dimensions of matrix A (n, m): 2 2
Enter row 1: 1 0
Enter row 2: 1 1
Enter dimensions of matrix B (m, p): 2 2
Enter row 1: 1 0
Enter row 2: 0 1
Matrix A:
 [[1 0]
 [1 1]]
Matrix B:
 [[1 0]
 [0 1]]
Crisp max-min composition:
 [[1 0]
 [1 1]]
\end{lstlisting}

\newpage
\section{Crisp composition using max-product}

\lstinputlisting[language=Python, caption=Crisp max-product composition]{crisp_max_product_composition.py}

\newpage
\begin{lstlisting}[caption=Output of Crisp max-product composition]
Enter dimensions of matrix A (n, m): 2 2
Enter row 1: 1 0
Enter row 2: 0 0
Enter dimensions of matrix B (m, p): 2 2
Enter row 1: 1 0
Enter row 2: 1 0
Matrix A:
 [[1 0]
 [0 0]]
Matrix B:
 [[1 0]
 [1 0]]
Crisp max-product composition:
 [[1 0]
 [0 0]]
\end{lstlisting}

\newpage
\section{Check whether a Fuzzy relation satisfies the equivalence property or not}

\lstinputlisting[language=Python, caption=Fuzzy equivalence]{fuzzy_equivalence_check.py}

\begin{lstlisting}[caption=Output of Check Fuzzy relation equivalence]
Enter dimensions of matrix R (n, m): 2 2
Enter row 1: 0.5 0.5   
Enter row 2: 0.5 0.5
Matrix R:
 [[0.5 0.5]
 [0.5 0.5]]
The Fuzzy relation satisfies the equivalence property.


Enter dimensions of matrix R (n, m): 2 2
Enter row 1: 0.6 0.5
Enter row 2: 0.8 0.1
Matrix R:
 [[0.6 0.5]
 [0.8 0.1]]
The Fuzzy relation does not satisfy the equivalence property.
\end{lstlisting}

\end{document}
